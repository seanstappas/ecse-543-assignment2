\documentclass[a4paper,titlepage]{article}
\usepackage[utf8]{inputenc}
\usepackage{fullpage}
\usepackage{indentfirst}
\usepackage[per-mode=symbol]{siunitx}
\usepackage{listings}
\usepackage{graphicx}
\usepackage{color}
\usepackage{amsmath}
\usepackage{array}
\usepackage[hidelinks]{hyperref}
\usepackage[format=plain,font=it]{caption}
\usepackage{subcaption}
\usepackage{standalone}
\usepackage[nottoc]{tocbibind}
\usepackage[noabbrev,capitalize,nameinlink]{cleveref}
\usepackage{listings}
\usepackage{titlesec}
\usepackage{minted}
\usepackage{booktabs}
\usepackage{csvsimple}
\usepackage{siunitx}
\usepackage[super]{nth}
\usepackage[titletoc]{appendix}

% Custom commands
\newcommand\numberthis{\addtocounter{equation}{1}\tag{\theequation}}
\newcommand{\code}[1]{\texttt{#1}}
\newcolumntype{P}[1]{>{\centering\arraybackslash}p{#1}}

\setminted{linenos,breaklines,fontsize=auto}

%\titleformat*{\section}{\normalsize\bfseries}
%\titleformat*{\subsection}{\small\bfseries}
\renewcommand{\thesubsection}{\thesection.\alph{subsection}}
\providecommand*{\listingautorefname}{Listing}
\newcommand*{\Appendixautorefname}{Appendix}

%opening
\title{\textbf{ECSE 543 \\ Assignment 2}}
\author{Sean Stappas \\ 260639512}
\date{November \nth{13}, 2017}

\begin{document}
	\sloppy
	\maketitle
	
	\tableofcontents
	
	
	\twocolumn
	
	\section*{Introduction}
	
	\section{Finite Element Triangles}
	
	The equation for the $\alpha$ parameter for a general vertex $i$ of a finite element triangle can be seen in \autoref{eq:alpha}, where $i+1$ and $i+2$ implicitly wraps around when exceeding 3.
	
	\begin{equation} \label{eq:alpha}
		\begin{split}
			\alpha_i(x, y) = \frac{1}{2A} \big[
			& (x_{i+1}y_{i+2} - x_{i+2}y_{i+1}) \\
			& + (y_{i+1} - y_{i+2})x \\
			& + (x_{i+2} - x_{i+1})y \big]
		\end{split}
	\end{equation}
	
	Using \autoref{eq:alpha}, we can solve for the entries of the local $S$ matrix, as shown in \autoref{eq:local_s}. This was used in the program to compute every entry for both example triangles.
	
	\begin{equation} \label{eq:local_s}
		\begin{split}
			S_{ij} 
			= &\int\displaylimits_{\Delta_e}\nabla\alpha_i\cdot\nabla\alpha_j dS \\
			= &\frac{1}{4A} \big[ (y_{i+1} - y_{i+2})(y_{j+1} - y_{j+2}) \\
			&+ (x_{i+2} - x_{i+1})(x_{j+2} - x_{j+1})
			\big]
		\end{split}
	\end{equation}
	
	The local $S$ matrix for the first triangle can be seen in \autoref{eq:local_s1}.
	
	\begin{equation} \label{eq:local_s1}
		S_1 =
			\begin{bmatrix}
				0.50 & -0.50 & 0.00 \\
				-0.50 & 1.00 & -0.50 \\
				0.00 & -0.50 & 0.50 
			\end{bmatrix}
	\end{equation}
	
	The local $S$ matrix for the second triangle can be seen in \autoref{eq:local_s2}.
	
	\begin{equation} \label{eq:local_s2}
		S_2 =
			\begin{bmatrix}
				1.00 & -0.50 & -0.50 \\
				-0.50 & 0.50 & 0.00 \\
				-0.50 & 0.00 & 0.50 
			\end{bmatrix}
	\end{equation}
	
	\section{Finite Element Coaxial Cable}
	
	\subsection{Mesh}
	
	\subsection{Electrostatic Potential}
	
	\subsection{Capacitance}
	
	The finite element energy equation can be seen in \autoref{eq:energy}.
	
	\begin{equation} \label{eq:energy}
		W = \frac{1}{2} U_{con}^T S U_{con}
	\end{equation}
	
	% TODO: Find correct equation here
	\begin{equation} \label{eq:energy_capacitor}
		W = \frac{1}{2} C V^2
	\end{equation}
	
	\section{Conjugate Gradient Coaxial Cable}
	
	\subsection{Positive Definite Test}
	
	\subsection{Matrix Solution}
	
	\subsection{Residual Norm}
	
	\subsection{Potential Comparison}
	
	\subsection{Capacitance Improvement}
	
%	\begin{figure}[!htb]
%		\centering
%		\includegraphics[width=0.5\columnwidth]{plots/q1_circuit_1.pdf}
%		\caption
%		{Test circuit 1 with labeled nodes.}
%		\label{fig:q1_circuit_1}
%	\end{figure}
%
%	\begin{table}[!htb]
%		\centering
%		\caption{Voltage at labeled nodes of circuit 1.}
%		\csvautobooktabular{csv/q1_circuit_1.csv}
%		\label{table:q1_circuit_1}
%	\end{table}
	
	\onecolumn
	
	\begin{appendices}
		
		\section{Code Listings} \label{appendix:code}
		
		\setminted{linenos,breaklines,fontsize=\footnotesize}
		
%		\begin{center}
%			\captionof{listing}{Custom matrix package (\texttt{matrices.py}).}
%			\inputminted{python}{../matrices.py}
%			\label{lst:matrices}
%		\end{center}
		
		\section{Output Logs} \label{appendix:logs}
		
%		\begin{center}
%			\captionof{listing}{Output of Question 1 program (\texttt{q1.txt}).}
%			\inputminted{pycon}{logs/q1.txt}
%			\label{lst:q1_log}
%		\end{center}

	\end{appendices}

\end{document}
